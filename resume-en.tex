%%
%% Copyright (c) 2018-2019 Weitian LI <wt@liwt.net>
%% CC BY 4.0 License
%%
%% Résumé
%% ------
%% A short document (1-2 pages) to sum up the job-related accomplishments
%% and experience.
%%
%% Checklist
%% ---------
%% * Contact Information
%% * Work History / Experience
%% * Education
%% * Skills
%% * Summary & Objective (optional)
%% * Hobbies & Interests (optional)
%%
%% Credits
%% -------
%% * CV vs. Resume: What is the Difference? When to Use Which?
%%   https://uptowork.com/blog/cv-vs-resume-difference
%% * How to Make a Resume: A Step-by-Step Guide (+30 Examples)
%%   https://uptowork.com/blog/how-to-make-a-resume
%% * Entry-Level Resume: Sample and Complete Guide (+20 Examples)
%%   https://uptowork.com/blog/entry-level-resume-example
%%
%% Created: 2018-04-14
%%

% English version
\documentclass{resume}

% Adjust icon size (default: same size as the text)
\iconsize{\Large}

% File information shown at the footer of the last page
\fileinfo{%
  \faCopyright{} 2025, Ziya Tang \hspace{0.5em}
  \creativecommons{by}{4.0} \hspace{0.5em}
  \githublink{tcztzy}{resume} \hspace{0.5em}
  \faEdit{} \today
}

\name{Ziya}{TANG}

\keywords{Python, C, Data Science, Agricultural System Modeling}

% \tagline{\icon{\faBinoculars}} <position-to-look-for>}
% \tagline{<current-position>}

% \photo{<height>}{<filename>}

\profile{
  \mobile{153-5770-8868}
  \email{tcztzy@hotmail.com}
  \github{tcztzy} \\
  \degree{PhD Candidate}
  \university{Nanjing Agricultural University (NAU)}
  \birthday{1992 Mar.}
  \address{Nanjing}
  % Custom information:
  % \icontext{<icon>}{<text>}
  % \iconlink{<icon>}{<link>}{<text>}
}

\begin{document}
\makeheader

%======================================================================
% Summary & Objectives
%======================================================================
% Master in Agricultural Engineering (electrification and automation)
% with good foundations of agronomy, math, and statistics.
% Proficient in data modeling and analysis,
% and enthusiastic about computer and network technologies.
% Skilled in Python, and C/C++ programming, and had 7 years of experience in Linux.
% Passionate about open source and share multiple projects on my
% \link{https://github.com/tcztzy}{GitHub}.
% Meanwhile, I am an active Python developer who maintained 6 packages in
% \link{https://pypi.org/user/tcztzy/}{Pypi} and a contributor to several other
% open-source projects.

%======================================================================
\sectionTitle{Competences \& Languages}{\faWrench}
%======================================================================
\begin{competences}[10em]
  \comptence{Operating Systems}{
    \icon{\faUbuntu} Ubuntu Linux (11 years)
    \icon{\faLinux} Arch Linux (9 years)
  }
  \comptence{Programming}{%
    Python, C/C++, Rust, R, MATLAB, Perl, \TeX
  }
  \comptence{Tools}{%
    Git, CMake, Visual Studio Code, Ansible, \LaTeX
  }
  \comptence{Data Analysis}{%
    Pandas, Matplotlib, Tikz
  }
  \comptence{Deep Learning}{%
    TensorFlow, PyTorch, Transformers, CUDA
  }
  \comptence{Large Language Model}{%
    SwarmX (multi-agent system), Omni WebUI (web UI), FastOAI (multi-source API integration)
  }
  \comptence{\icon{\faLanguage} Languages}{
    \textbf{English} ---
      reading \& writing (good);
      listening \& speaking (conversant)
      CET-6 456
  }
\end{competences}

%======================================================================
\sectionTitle{Education}{\faGraduationCap}
%======================================================================
\begin{educations}
  \education%
    {2022.09}%
    [est. 2026.06]%
    {Nanjing Agricultural University}%
    {Academy for Advanced Interdisciplinary Studies}%
    {Agricultural Engineering}%
    {PhD Candidate}

  \separator{0.5ex}
  \education%
    {September 2019}%
    [September 2022]%
    {Tarim University}%
    {College of Information Engineering}%
    {Agricultural Engineering}%
    {Master's Degree}

  \separator{0.5ex}
  \education%
    {September 2010}%
    [June 2014]%
    {Northwest A\&F University}%
    {College of Plant Protection}%
    {Plant Protection}%
    {Bachelor's Degree}
\end{educations}

%======================================================================
\sectionTitle{Computer Skills}{\faCogs}
%======================================================================
\begin{itemize}
  \item Designed and developed the \link{https://www.xfyun.cn/solutions/big-data-platform}{Odeon Big Data Platform} (Django, Vue, Ansible, Hadoop)
  \item Good at Microsoft Azure cloud computing platform's architecture and development, helped customers from worldwide migrate their business to cloud
  \item Good at DevOps for Linux operating system, containerized some scientific computation softwares
\end{itemize}

\sectionTitle{Scientific Publications}{\faFlask}
\begin{itemize}
  \item Tang, Ziya (2022), Simulation of non{-}mulched cultivated cotton growth in saline areas of South Xinjiang[D].Tarim University, 2022.DOI:10.27708/d.cnki.gtlmd.2022.000156.
  \item Tang, Ziya (2022), Modeling Non-Mulched Cultivation Cotton Growth and Yield Responses to Irrigation Scheduling Using Canopy-Modified Cotton2k Model.
\end{itemize}

%======================================================================
\sectionTitle{Research Achievements}{\faAtom}
%======================================================================
\begin{itemize}
  \item Developed a suite of utilities to semi-automate the
    Refactored original Cotton2K simulation model
    \link{https://github.com/tcztzy/cottonmodel}{\texttt{cottonmodel}}.
  \item Published 2 co-authored papers.
\end{itemize}

%======================================================================
\sectionTitle{Career}{\faBriefcase}
%======================================================================
\begin{experiences}
  \experience%
    [March 2017]%
    {March 2018}%
    {Senior Software Engineer @ iFLYTEK Corporation}%
    [\begin{itemize}
      \item Developed \link{https://www.xfyun.cn/solutions/big-data-platform}{Odeon Big Data Platform}, implemented backend featured modules and visualized in web browser.
    \end{itemize}]

  \separator{0.5ex}
  \experience%
    [July 2016]%
    {March 2017}%
    {Support Engineer @ \icon{\faMicrosoft} Microsoft China}%
    [\begin{itemize}
      \item Technical supporting for Microsoft Azure
      \item Technical supporting for Office 365 (Microsoft 365)
    \end{itemize}]
\end{experiences}

\end{document}
