%%
%% Copyright (c) 2018-2019 Weitian LI <wt@liwt.net>
%% CC BY 4.0 License
%%
%% Created: 2018-04-11
%%

% Chinese version
\documentclass[zh]{resume}

% Adjust icon size (default: same size as the text)
\iconsize{\Large}

% File information shown at the footer of the last page
\fileinfo{%
  \faCopyright{} 2021, Ziya TANG \hspace{0.5em}
  \creativecommons{by}{4.0} \hspace{0.5em}
  \githublink{tcztzy}{resume} \hspace{0.5em}
  \faEdit{} \today
}

\name{梓涯}{唐}

\keywords{Python, C, Data Science, Agricultural System Modeling}

% \tagline{\icon{\faBinoculars}} <position-to-look-for>}
% \tagline{<current-position>}

% \photo{<height>}{<filename>}

\profile{
  \mobile{153-5770-8868}
  \email{tcztzy@hotmail.com}
  \github{tcztzy} \\
  \university{塔里木大学}
  \degree{农业工程 \textbullet 硕士}
  \birthday{1992-03-13}
  \address{阿拉尔}
  % Custom information:
  % \icontext{<icon>}{<text>}
  % \iconlink{<icon>}{<link>}{<text>}
}

\begin{document}
\makeheader

%======================================================================
% Summary & Objectives
%======================================================================
{\onehalfspacing\hspace{2em}%
农业工程专业(农业电气化与自动化方向)硕士研究生,有扎实的农学、数学与统计学基础,
擅长数据建模与分析,热衷计算机和网络技术,熟练掌握 Python 和 C/C++ 语言编程,
有 8 年的 Linux 使用经验。
积极实践自由开源精神,
在 \link{https://github.com/tcztzy}{GitHub} 上分享多个项目,
在 \link{https://pypi.org/user/tcztzy/}{PyPi} 上维护 6 个包,
并积极参与其他多个开源项目。
\par}

%======================================================================
\sectionTitle{技能和语言}{\faWrench}
%======================================================================
\begin{competences}
  \comptence{操作系统}{%
    \icon{\faUbuntu} Ubuntu Linux (8 年)
    \icon{\faLinux} Arch Linux (6 年)
  }
  \comptence{编程}{%
    \icon{\faPython} Python, \icon{\faCode} C/C++, \icon{\faRProject} R, \TeX
  }
  \comptence{工具}{%
    Git, CMake, Meson, Visual Studio Code, \LaTeX
  }
  \comptence{数据分析}{%
    Pandas; Matplotlib, ggplot2, Tikz, Scikit-learn, TensorFlow
  }
  \comptence{网站开发}{%
    Flask, Django, Express.js, React.js
  }
  \comptence{\icon{\faLanguage} 语言}{
    \textbf{英语} --- 读写(优良),听说(日常交流)CET-6 456
  }
\end{competences}

%======================================================================
\sectionTitle{教育背景}{\faGraduationCap}
%======================================================================
\begin{educations}
  \education%
    {2019.09}%
    [2022.09]%
    {塔里木大学}%
    {信息工程学院}%
    {农业工程}%
    {硕士}

  \separator{0.5ex}
  \education%
    {2010.09}%
    [2014.06]%
    {西北农林科技大学}%
    {植物保护学院}%
    {植物保护}%
    {学士}
\end{educations}

% %======================================================================
% \sectionTitle{计算机技能}{\faCogs}
% %======================================================================
% \begin{itemize}
%   \item 参与设计并开发了科大讯飞 \link{https://www.xfyun.cn/solutions/big-data-platform}{Odeon 大数据平台}
%     (Django, Vue, Ansible, Hadoop)
%   \item 掌握 Microsoft Azure 云计算平台架构和开发,协助全球范围内的客户完成企业上云的工作
%   \item 熟悉 Linux 操作系统运维,将部分科研行业软件容器化
% \end{itemize}

%======================================================================
\sectionTitle{个人项目}{\faCode}
%======================================================================
\begin{itemize}
  \item \link{https://github.com/tcztzy/cotton2k}{\texttt{cotton2k}}:
    (C++, Rust, Python)
    棉花生长模拟模型
  \item \link{https://github.com/tcztzy/zac}{\texttt{zac}}:
    (Python)
    Zac 大气校正工具,可校正Landsat、Sentinel和MODIS系列卫星,基于\link{https://github.com/mutiply-org/SAIC}{\texttt{SAIC}}
  \item \link{https://github.com/tcztzy/uyghur}{\texttt{uyghur}}:
    (Python, Perl) 现行维吾尔文和拉丁维吾尔文转换工具
  \item \link{https://github.com/tcztzy/resume}{\texttt{resume}}:
    (\LaTeX)
    \emph{此简历}的模板和源文件
\end{itemize}

%======================================================================
\sectionTitle{科研竞赛}{\faAtom}
%======================================================================
\begin{itemize}
  \item 2019 年“华为杯”第十六届中国研究生数学建模竞赛国家三等奖。
  \item 2021 年“华为杯”第十八届中国研究生数学建模竞赛国家三等奖。
  \item 2021 年“蓝桥杯”第十二届蓝桥杯大赛个人赛软件类全国总决赛 Python 大学生组三等奖。
\end{itemize}

%======================================================================
\sectionTitle{工作经历}{\faBriefcase}
%======================================================================
\begin{experiences}
  \experience%
    [2017.03]%
    {2018.03}%
    {软件工程师 @ 科大讯飞股份有限公司}%
    [\begin{itemize}
      \item \link{https://www.xfyun.cn/solutions/big-data-platform}{Odeon 大数据平台}的开发,主要负责后端功能模块的构建和前端页面的展示。
    \end{itemize}]

  \separator{0.5ex}
  \experience%
    [2016.07]%
    {2017.03}%
    {技术支持工程师 @ \icon{\faMicrosoft}微软中国}%
    [\begin{itemize}
      \item Microsoft Azure 云计算平台的开发技术支持
      \item Office 365(现Microsoft 365)的开发技术支持
    \end{itemize}]
  
  \separator{0.5ex}
  \experience%
    [2015.07]%
    {2017.03}%
    {软件工程师 @ 南京振古科技有限公司(初创公司)}%
    [\begin{itemize}
      \item 后端开发(Django),完成用户注册、数据存储和搜索等功能
      \item Docker 容器开发,将科研行业软件容器化
    \end{itemize}]
  
  % \separator{0.5ex}
  % \experience%
  %   [2014.07]%
  %   {2015.07}%
  %   {985村官计划村官 @ 江苏省宿迁市}%
  %   [\begin{itemize}
  %     \item 完成镇办公室日常事务性工作
  %   \end{itemize}]
\end{experiences}

\end{document}
